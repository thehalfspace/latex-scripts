\documentclass[11pt]{article}
\usepackage[utf8]{inputenc}
\usepackage{graphicx}
\usepackage{subfig}
\usepackage{enumerate}
\usepackage{amsmath}
\usepackage[letter,width=150mm,top=30mm,bottom=30mm]{geometry}
\usepackage[font={footnotesize}]{caption}
\usepackage[labelfont=bf]{caption}
\usepackage[english]{babel}
\usepackage{natbib}
\setlength{\parskip}{1em}
% \renewcommand{\baselinestretch}{2}
\DeclareMathOperator{\arcsinh}{arcsinh}
\usepackage{setspace}
% \renewcommand{\theenumi}{\roman{enumi}}
\graphicspath{{../images/}}
     
\begin{document}
\pagenumbering{gobble}
\title{\textbf{Constraints on depths and magnitude-frequency distribution of simulated earthquakes on strike-slip faults surrounded by damaged zones}}
\author{Prithvi Thakur\\University of Michigan}

\maketitle

\doublespacing
\section*{Abstract}
Observations of earthquakes on major strike-slip faults (e.g. San-Andreas fault system) show that the seismicity is highly concentrated along a fault-parallel narrow zone of damaged rocks, referred to as damaged fault zone. However, most observations are limited to timescales of years to decades, and the long-term influence of fault zone damage on complete earthquake cycles, i.e., hundreds of years, is not well understood. By simulating earthquake sequences for a thousand year timescale, we show that the geometry of the damaged fault zone can have pronounced effect on the location of earthquake hypocenters and the earthquakes are more likely to nucleate within the damaged fault zone. We also show that damaged fault-zone like material heterogeneity can induce a power-law behavior between the cumulative number of earthquakes and their magnitudes. These results suggest that the understanding of damaged fault zone geometry and evolution through time can help us constrain the depth of earthquakes on a fault and provide a better understanding of the physics of long-term earthquake sequences.
\end{document}
