\documentclass{article}
\usepackage[utf8]{inputenc}
\usepackage{graphicx}
\usepackage{enumerate}
\usepackage{amsmath}

\graphicspath{{images/}}

\setlength{\parindent}{4em}
\setlength{\parskip}{1em}
\renewcommand{\baselinestretch}{1.3}

\makeatletter
% we use \prefix@<level> only if it is defined
\renewcommand{\seccntformat}[1]{%
  \ifcsname prefix@#1\endcsname
    \csname prefix@#1\endcsname
  \else
    \csname the#1\endcsname\quad
  \fi}
% define \prefix@section
\newcommand\prefix@section{Problem \thesection: }
\makeatother

\title{Spectral Methods: Final Exam}
\author{Prithvi Thakur}
\date{April 22, 2018}

\begin{document}

\maketitle

% Problem 1
\section{}
The given asymptotic form implies that the function is singular at the endpoints, but it may still be infinitely-differentiable.

% Problem 2
\section{}
\includegraphics[scale=0.25]{1a.png}
\renewcommand{\theenumi}{\roman{enumi}}
\begin{enumerate}
\item Graph II shows spectral blocking. This occurs due to numerical instability and noise at higher wavenumbers, and due to aliasing. This results in increase in the spectral coefficients at higher wavenumbers, which is a numerical artifact and is unphysical. This can be treated by increasing resolution or by decreasing the timestep to stay within the CFL limit.
    
\item Graph I shows coefficients erroneous due to round off errors. The reason for round off errors is the machine precision. After a certain decimal precision($10^{-16}$ in this case), the coefficients keep oscillating around that decimal place in a random fashion.

\item Graph I, III, IV shows coefficients of the program working correctly. Roundoff errors will be seen in almost every problem with such small coefficients. Graph IV shows coefficients with typical geometric convergence, with oscillatory coefficients. Graph III shows a typical geometric convergence which later slows down to algebraic convergence. All these rates of convergence are mostly dictated by the singularities in the domain. Graph II shows spectral blocking, where the coefficients increase after a certain degree. This gives spurious numerical artifacts, and should be treated by either increasing resolution or decreasing the timestep according to CFL stability limit.

\item  All the graphs may be different depending on whether the basis is Fourier or Chebyshev. The rate of convergence depends on the location of the gravest singularity. Therefore, if the domain is periodoc, Fourier will have a much faster convergence than Chebyshev, and vice versa for non-periodic smooth functions.
\end{enumerate}


% Probelm 3
\section{}
The optimum choice of basis set is dependent on the geometry of the problem. In most of the cases, Fourier methods and Chebyshev methods (which are just the Fourier cosine series with a change of coordinate) are most optimum choice for a spectral model. There are multiple reasons for choosing either of the above as basis sets. Firstly, \textbf{Fourier series are period}, and are a perfect choice of basis sets for periodic functions. This enables us to compute the solution over infinite domain using periodicity. Secondly, the \textbf{computation of derivatives is very fast and efficient with the use of Fast Fourier Transform algorithms}. For non-periodic functions, Chebyshev series are a perfect choice. Since they are Fourier series in disguise, the computation and convergence is just as efficient. Another advantage of using the Chebyshev series basis set is that the collocation points are non-uniform, and the collocation grid is denser at the boundaries. This is an excellent approach to \textbf{tackle the Runge's phenomenon} where spectral errors are amplified at the endpoints. Thus they have \textbf{excellent convergence at endpoints}. For a small number of collocation points, the solution is \textbf{as good as analytic}. Both Fourier and Chebyshev series are known to \textbf{converge globally}, and the rate of convergence is only dictated by the singularities in the functions.


But there may be cases where both Fourier and Chebyshev polynomials are not the best choice of basis sets. Most of these cases are dictated by the geometry of the problem. There is a famous joke about this: "while 10 terms in the Fourier series is more than enough for any engineering application, even 30 terms of Fourier series expansion of drawing of a cow would not please a third grade art teacher". This just emphasizes the fact that Fourier series are not good for complicated geometry. For such cases, Spline interpolation would be much better choice. On the \textbf{surface of a sphere}, it is more efficient to use Spherical Harmonics than Chebyshev or Fourier series. Similarly, for \textbf{non periodic functions with infinite or semi-infinite domain}, rational Chebyshev functions should be used, as they are tailored to that domain. Another example is \textbf{Spectral Element methods, where Chebyshev polynomials perform poorly in comparison to Legendre collocation}. This is because it is much easier to perform Gauss-Lobatto-Legendre quadrature for spatial integration, and this advantage outshines any other weaknesses of the Legendre collocation in Spectral Element Methods.


% Problem 4
\section{}
\begin{enumerate}
    \item The isolines of stream functions are smooth in space, and the energy is constant in time, which implies that the total energy of the model is conserved. This indicates that there is no aliasing in the model, and a blow-up will not be observed. There are two viewpoints on the accuracy of spectral model using energy-conservation scheme: (1) Optimistic viewpoint argues that energy-conservation schemes are interpolation with physics constraints. These schemes, therefore, imply that the function is constrained by physics of the system. These constraints would therefore lead to an improved approximation of the real physics. (2) Pessimistic viewpoint argues that despite the numerical solution remaining bounded, there is no guarantee of its accuracy. For example, a constant discretized energy obviously excludes supersonic winds in the model. Smooth streamlines indicate that resolution of the model is sufficient to resolve the smallest feature in the system. Any numerical oscillations, implying a non-smooth streamline, would suggest that there are some features in the model not being observed at the current resolution.
    
    \item The most direct way to test the accuracy of the model is to decrease the timestep and spatial grid size, and if the model is smooth and unchanging for more refined spatial grid and smaller timestep, then we can reasonably assume that the model is accurate.  Another  way is to look at the convergence of spectral coefficients. This would help us in identifying spectral blocking, if observed. Shock-resolution, i.e., special finite difference or spectral algorithms to cope with fronts and shockwaves can be used as a remedy to the aliasing. A third way would be to use Orzag's two-third rule, i.e., filter the high wavenumbers so as to retain on two-thirds unfiltered wavenumbers. This filtered solution can be compared to the original solution, and if both are equally smooth, the solutions are accurate.
\end{enumerate}

% Problem 5
\section{}
General solution to the partial differential equation:

$$
u = f(x)\{Asin(100t) + Be^{-t}cos(t)\}
$$

\begin{enumerate}
    \item The fast part of the solution is $f(x)Asin(100t)$, and the slow part of the solution is $f(x)Be^{-t}cos(t)$.
    
    \item The motion on the slow manifold would be the slow part of the solution, i.e., $f(x)Be^{-t}cos(t)$. This is slow exponential decaying motion.
    
    \item If $A = 0$, the solution only has a slow part, therefore larger timesteps would be efficient. This can be achieved by implicit time-marching schemes, like the Crank Nicholson scheme or the Backward Euler scheme. The backward Euler would be better for larger timesteps and higher spatial resolutions, since it damps high frequencies and therefore is stable and immune to oscillations. For a given initial slow manifold condition, the Crank Nicholson scheme would be unconditionally stable.
    
    \item If $B = 0$, the solution only has a fast part, therefore explicit time marching scheme would be more efficient. For $B = 0$, the solution is periodic for a very small time period, therefore explicit method would not be too expensive.
    
    \item If $A = B$, we have both fast and slow parts in the solution. In this case, a semi-implicit method would be efficient, with slight compromise on the algorithm time, but permitting a longer timestep. 
\end{enumerate}


\section{}

Given partial differential equation:

$ u_{xx} + u_{yy} = f(x,y); \ u = 0$ on all sides of the unit square. 

To solve this Poisson's equation with homogeneous Dirichlet boundary conditions, we would solve an equivalent homogeneous Laplace equation and use a linear combination of the solution as a basis set for the Poisson's equation.

The equivalent Laplace equation is,
$$
u_{xx} + u_{yy} = 0; u(\pm1, :) = u(:, \pm1) = 0
$$
Using separation of variables, we get $u(x,y) = X(x)Y(y)$, thus,
\begin{align}
    u_{xx} = X"(x)Y(y);\ u_{yy} &= X(x)Y"(y) \\
    \implies X"Y + XY" &= 0 \\
    \implies \frac{X"}{X} = -\frac{Y"}{Y} &= k (const) \\ 
\end{align}

We can write this as two separate ODEs as follows,
\begin{align}
    X" - kX = 0;\ X(\pm1) = 0 \\
    Y" + kY = 0;\ Y(\pm1) = 0
\end{align}

The solution to the above Sturm Liouville Problems at their boundaries yield,
\begin{align}
    X(x) = A_1cosh(\sqrt{k}x) \\
    Y(y) = A_2cos(\sqrt{k}y)
\end{align}

Each of the above equations will have $\frac{N}{2}$ basis sets, for $N$ term expansion of the original equation.

Then the basis set for the original Poisson's equation would be a linear combination of the above:
\begin{align}
    u(x,y) = \sum_{n = 0}^{N/2}a_{n}cos((n+1)\pi x)cosh((n+1)\pi x)
\end{align}

% Problem 7
\section{}
Given nonlinear ODE,
\begin{align}
    u_{xx} + \lambda e^u = 0;\ u(\pm1) = 0
\end{align}

Since the boundary conditions are already homogeneous, and the Chebyshev series $T_{2n}(\pm1) = 1$, we can choose a basis set to be $T_{2n}(x) - 1$. For a one point collocation(N = 1), this would give us
\begin{align}
    u(x) &= A_o\phi(x);\ \phi(x) = T_2(x) - 1 \\
    \implies u(x) &= A_o(2x^2 - 1 - 1) \\
    \implies u(x) &= A(x^2 - 1);\ A = 2A_o
\end{align}

Substituting this solution in the original equation, we can calculate the residual R,
\begin{align}
R(x, A) = 2A + \lambda e^{A(x^2 -1)}
\end{align}

In the pseudospectral method, this residual should be zero at the collocation point $x_o$:
\begin{align}
    2A + \lambda e^{A(x_o^2 -1)} = 0
\end{align}

To compute the maximum value at $x_o = 0$, using equations (13) and (15)
\begin{align}
    u(x_o = 0; A) = -A \\
    2A + \lambda e^{-A} = 0 \\
\end{align}

Solving the above two equations, we obtain
\begin{align}
    -2u + \lambda e^u &= 0 \\
    \implies \lambda e^u - 2u &= 0
\end{align}


% Problem 8
\section{}
Given the PDE with homogeneous Dirichlet boundaries at,
\begin{align}
    u_{zz} + u_{yy} - u &= f(z,y) \\
    y = 1 + cos(z/2);\ y = -1 - cos(z/2);\ y &= \pm1,\ z\in[-\pi, \pi]
\end{align}

We use the change of coordinates to scale z to [-1,1], 
\begin{align}
    x &= sin(z/2) \\
    \implies u_{zz} &= \frac{-1}{4}sin(z/2)u_{xx} \\
    \implies \frac{-1}{4}xu_{xx} + u_{yy} - u &= g(x,y);\ g(x,y) = h(x)f(z,y)
\end{align}

Now, our problem is in the unit square domain $[-1,1]\times[-1,1]$. We choose a tensor product basis set $\phi_{mn}(x,y) = \phi_m(x)\phi_n(y)$ for the domain $m = 1,...,Nx$ and $n = 1,...,Ny$. Since the boundary conditions are homogeneous, the boundary conditions would be implicitly fulfilled in the basis recombination of the functions. The interpolation grid would be a tensor product of the $[Nx,Ny]$ interpolation points. We store these $[Nx,Ny]$ tensors as a large 1D vectors with indices $[1, Nx.Ny]$ using a connectivity matrix.

\begin{align}
    u(x,y) = \sum_{m,n=0}^{N}a_{mn}T_m(x)T)n(y)\\
    r_k = <x_i, y_j>;\ i = 1,..,Nx,\ j = 1,..,Ny,\ k = 1,..,NxNy
\end{align}

Equation (26) is the set of basis functions at collocation points given by equation (27). We can then proceed with the pseudospectral discretization into matrix equation to solve the PDE.
\end{document}

