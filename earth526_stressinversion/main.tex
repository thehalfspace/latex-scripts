%%%%%%%%%%%%%%%%%%%%%%%%%%%%%%%%%%%%%%%%%%%%%%%%%%%%%%%%
%%%%%%%%%%%%%%%%%%%%%%%%%%%%%%%%%%%%%%%%%%%%%%%%%%%%%%%%
%  PREAMBLE : USE AT THE BEGINNING OF EVERY PRESENTATION 
%%%%%%%%%%%%%%%%%%%%%%%%%%%%%%%%%%%%%%%%%%%%%%%%%%%%%%%%
%%%%%%%%%%%%%%%%%%%%%%%%%%%%%%%%%%%%%%%%%%%%%%%%%%%%%%%%
\documentclass[11pt]{beamer}

\usepackage{appendixnumberbeamer}

\usepackage{booktabs}
\usepackage[scale=2]{ccicons}

\usepackage{pgfplots}
\usepgfplotslibrary{dateplot}

\usepackage{fontspec}
\defaultfontfeatures{Ligatures=TeX}
\setsansfont{Roboto}[
    Path = /usr/local/texlive/2017/texmf-dist/fonts/truetype/google/roboto/,
    Extension =.ttf,
    ItalicFont = *-RegularItalic,
    BoldFont = *-Black,
    UprightFont = *-Regular ]

% set beamer fonts
\setbeamerfont{title}{size=\Huge}
% \setbeamerfont{itemize/enumerate body}{size=\large}
% \setbeamerfont{itemize/enumerate subbody}{size=\normalsize}
% \setbeamerfont{itemize/enumerate subsubbody}{size=\small}
\setbeamerfont{frametitle}{size=\huge}

% Change default title color
\usepackage{xcolor}
\definecolor{myblue}{RGB}{18, 65, 104}  % title color
\definecolor{myhl}{RGB}{21, 126, 186}   % texthighlight color
\definecolor{myblack}{RGB}{43, 40, 40}
\setbeamercolor{titlelike}{parent=structure,fg=myblue}

% Chaneg bullet points
% \setbeamertemplate{itemize items}{\circle}
\setbeamercolor*{item}{fg=myblue}

% disable navigation
\setbeamertemplate{navigation symbols}{}

% disable "Figure:" in the captions
\setbeamertemplate{caption}{\tiny\insertcaption}
\setbeamertemplate{caption label separator}{}

% set new command for code/highlight
\newcommand{\code}[1]{\texttt{#1}}
\newcommand{\hl}[1]{\textcolor{myhl}{#1}}

%%%%%%%%%%%%%%%%%%%%%%%%%%%%%%%%%%%%%%%%%%%%%%%%%%%%%%%%

%----------------
% BEGIN DOCUMENT
%----------------
\title{\textbf{Slip on Faults and Stress Inversion}}
\date{Jan 30, 2019}
\author{Prithvi Thakur}

\begin{document}

\maketitle

\begin{frame}{\textbf{Anderson's Theory of Faulting}}
    \begin{itemize}
        \item Earth's free surface cannot support shear, therefore, one of the principal stresses has to be vertical. \hl{Normal, Reverse, Strike-Slip Faulting}.
    \end{itemize}
\end{frame}

\begin{frame}{\textbf{Anderson's Theory of Faulting}}
    \begin{itemize}
        \item Limitations: Only works for intact rocks near the free surface. 
        \item More realistic faulting style: oblique-slip faulting. These occur on pre-existing weakness planes and at a depth, therefore none of the principal stresses are truly vertical.
        \item \hl{Pre-existing weakness:} Reactivated faults, bedding planes, weak layer (shale, salt), foliations, anisotropy in materials, etc.
    \end{itemize}
\end{frame}

\end{document}
